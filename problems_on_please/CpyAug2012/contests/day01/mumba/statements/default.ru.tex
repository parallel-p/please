Слова в языке Мумба-Юмба могут состоять только из букв {\bf a}, {\bf b}, {\bf c} и при этом:

\begin{itemize}
\item никогда не содержат двух букв {\bf b} подряд,
\item ни в одном слове никогда не встречается три одинаковых подслова подряд.
\end{itemize}

Например, по этому правилу в язык Мумба-Юмба не могут входить слова {\bf aaa}
(так как три раза подряд содержит подслово a), {\bf ababab} (так как три раза
подряд содержит подслово {\bf ab}), {\bf aabcabcabca} (три раза подряд содержит подслово {\bf abc}).
Все слова, удовлетворяющие вышеописанным правилам, входят в язык Мумба-Юмба.

Напишите программу, которая по данному слову определит, принадлежит ли оно этому языку.

\InputFile
Вводится одно слово, состоящее только из строчных букв {\bf a}, {\bf b}, {\bf c},
длины не более~100.

\OutputFile
Если слово входит в язык Мумба-Юмба, выведите YES, в противном случае выведите NO.
