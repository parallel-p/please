%Please contest template file
\documentclass[12pt,a4paper,oneside,twocolumn,landscape]{article}

\usepackage[T2A]{fontenc}
\usepackage[utf8]{inputenc}
\usepackage[english,russian]{babel}
\usepackage[russian,landscape]{olymp}
\usepackage{graphicx}
\DeclareGraphicsRule{*}{mps}{*}{}
\graphicspath{{../statements/}}
\usepackage{amsmath,amssymb,amsthm}
\usepackage{epsfig}
\usepackage{import}
\usepackage{wrapfig}
\usepackage{epigraph}

\contest
{Берендеевы Поляны}
{Параллель C.py - день 5}
{2 августа 2012}


\renewcommand{\t}[1]{\ifmmode{\mathtt{#1}}\else{\texttt{#1}}\fi}
\renewcommand{\le}{\leqslant}
\renewcommand{\ge}{\geqslant}
\newcommand{\bs}{\mbox{$\backslash$}}

\begin{document}

\raggedbottom

\begin{problem}{Плавные числа}{numbers.in}{numbers.out}{1 секунда}{64 мегабайта}{A}
\graphicspath{{../numbers/statements/}}


Назовём натуральное число плавным, если разность любых двух его соседних цифр 
не превосходит по модулю единицы. Вам необходимо определить количество $N$-значных 
плавных чисел.

\InputFile
В единственной строке входного файла одно число $N$ ($1 \leqslant N \leqslant 20$).

\OutputFile
Вывести одно число~--- искомое количество плавных чисел.


\Example

\begin{example}
\exmp{
2
}{26
}%
\end{example}


\end{problem}

\bigskip\bigskip
\begin{problem}{Наибольшая последовательнократная подпоследовательность}{sequence.in}{sequence.out}{1 секунда}{64 мегабайта}{B}
\graphicspath{{../sequence/statements/}}


Для заданной числовой последовательности $a_1, a_2, \dots , a_n$ требуется найти длину максимальной последовательнократной подпоследовательности.

Для последовательнократной подпоследовательности $a_{k_1}, a_{k_2}, \dots, a_{k_t}$ ($k_1 < k_2 < \dots < k_t$) верно, что $a_{k_i} | a_{k_j}$ при $1 \leqslant i < j \leqslant t$ (утверждение <<$a|b$>> эквивалентно <<$b$ кратно $a$>>). Подпоследовательность из одного элемента полагается последовательнократной по определению.

\InputFile
В первой строке входного файла записаны $N$ натуральных чисел ($1 \leqslant N \leqslant 1000$), не превосходящих $2\cdot10^9$~--- последовательность.

\OutputFile
Вывести единственное число, равное длине максимальной последовательнократной подпоследовательности.


\Example

\begin{example}
\exmp{
3 6 5 12
}{3
}%
\end{example}


\end{problem}

\bigskip\bigskip
\begin{problem}{Наибольшая общая подпоследовательность}{lcs.in}{lcs.out}{2 секунды}{64 мегабайта}{C}
\graphicspath{{../largest-common-subsequence/statements/}}


Даны две последовательности. Найдите длину их наибольшей общей подпоследовательности (подпоследовательность~--- это то, что можно получить из данной последовательности вычеркиванием некоторых элементов).

\InputFile
В первой строке входного файла через пробел записаны $N$ членов первой последовательности ($1 \leqslant N \leqslant 1000$)~---
 целых чисел, не превосходящих 10\,000 по модулю.
Во второй строке через пробел записаны $M$ членов второй последовательности ($1 \leqslant M \leqslant 1000$)~--- 
целые числа, не превосходящие 10\,000 по модулю.

\OutputFile
В первую строку выходного файла требуется вывести единственное целое число: длину наибольшей общей подпоследовательности или число 0, если такой не существует.
Во вторую строку выходного файла требуется вывести саму наибольшую общую подпоследовательность, 
через пробел (если подпоследовательностей несколько, выведите любую).


\Example

\begin{example}
\exmp{
1 2 3
2 1 3 5
}{2
2 3
}%
\end{example}


\end{problem}

\bigskip\bigskip
\begin{problem}{Покупка билетов}{tickets.in}{tickets.out}{1 секунда}{64 мегабайта}{D}
\graphicspath{{../tickets/statements/}}


За билетами на премьеру нового мюзикла выстроилась очередь из $N$ человек, каждый из которых хочет купить $1$ билет. На всю очередь работала только одна касса, поэтому продажа билетов шла очень медленно, приводя <<постояльцев>> очереди в отчаяние. 
Самые сообразительные быстро заметили, что, как правило, несколько билетов в одни руки кассир продаёт быстрее, чем когда эти же билеты продаются по одному. Поэтому они предложили нескольким подряд стоящим людям отдавать деньги первому из них, чтобы он купил билеты на всех. 

Однако для борьбы со спекулянтами кассир продавала не более $3$-х билетов в одни руки, поэтому договориться таким образом между собой могли лишь $2$ или $3$ подряд стоящих человека.

Известно, что на продажу $i$-му человеку из очереди одного билета кассир тратит $A_i$ секунд, на продажу двух билетов~--- $B_i$ секунд, трех билетов~--- $C_i$ секунд. Напишите программу, которая подсчитает минимальное время, за которое могли быть обслужены все покупатели.

Обратите внимание, что билеты на группу объединившихся людей всегда покупает первый из них. Также никто в целях ускорения не покупает лишних билетов (то есть билетов, которые никому не нужны).

\InputFile
Во входном файле записано $N$ троек натуральных чисел $A_i$, $B_i$, $C_i$ ($1 \leqslant N \leqslant 5000$). Каждое из этих чисел не превышает $3600$.
 Люди в очереди нумеруются начиная от кассы.

\OutputFile
В выходной файл выведите одно число~--- минимальное время в секундах, за которое могли быть обслужены все покупатели.


\Example

\begin{example}
\exmp{
5 10 15
2 10 15
5 5 5
20 20 1
20 1 1
}{12
}%
\end{example}


\end{problem}

\bigskip\bigskip
\begin{problem}{Рюкзак}{knapsack.in}{knapsack.out}{2 секунды}{64 мегабайта}{E}
\graphicspath{{../knapsack/statements/}}


Найдите максимальный вес золота, который можно унести в рюкзаке вместительностью $S$, если есть $N$ золотых слитков с заданными весами.



\InputFile
В первой строке входного файла запиано одно число~--- $S$ ($1 \leqslant S \leqslant 10\,000$). 

Далее следует $N$ неотрицательных целых чисел ($1 \leqslant N \leqslant 300$), не превосходящих 100\,000~--- веса слитков.

\OutputFile
Выведите искомый максимальный вес.


\Examples

\begin{example}
\exmp{
10
1 4 8
}{9
}%
\exmp{
20
5 7 12 18
}{19
}%
\end{example}


\end{problem}

\bigskip\bigskip
\begin{problem}{Рюкзак с массами}{knapsack2.in}{knapsack2.out}{1 секунда}{64 мегабайта}{F}
\graphicspath{{../knapsack2/statements/}}
Дано $N$ предметов массой $m_1$, $\dots$, $m_N$ и стоимостью $c_1$, $\dots$, $c_N$ 
соответственно.

Ими наполняют рюкзак, который выдерживает вес не более $M$. 
Определите набор предметов, который можно унести в рюкзаке, имеющий наибольшую стоимость.

\InputFile
В первой строке вводится натуральное число M, не превышающее $10\,000$.

Во второй строке вводятся $N$ ($N \le 100$) натуральных чисел $m_i$, не превышающих 100.

В третьей строке вводятся $N$ натуральных чисел $c_i$, не превышающих 100.

\OutputFile
Выведите номера предметов (числа от 1 до $N$), 
которые войдут в рюкзак наибольшей стоимости.


\Example

\begin{example}
\exmp{
6
2 4 1 2
7 2 5 1
}{1
3
4
}%
\end{example}


\end{problem}


\end{document}
