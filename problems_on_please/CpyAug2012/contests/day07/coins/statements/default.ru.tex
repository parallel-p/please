% Statements
В Волшебной стране используются монетки достоинством $A_1, A_2, \dots , A_M$. Волшебный человечек пришел в магазин и обнаружил, что у него есть ровно по две монетки каждого достоинства. Ему нужно заплатить сумму $N$. Напишите программу, определяющую, сможет ли он расплатиться без сдачи.

\InputFile 
Сначала вводится число $N (1 \leq N \leq 10^9)$, затем — число $M (1 \leq M \leq 10)$ и далее $M$ попарно различных чисел $A_1, A_2, \dots , A_M (1 \leq A_i \leq 10^9)$.

\OutputFile
Выведите сначала $K$ — количество монет, которое придется отдать Волшебному человечку, если он сможет заплатить указанную сумму без сдачи. Далее выведите $K$ чисел, задающих достоинства монет. Если решений несколько, выведите вариант, в котором Волшебный человек отдаст наименьшее возможное количество монет. Если таких вариантов несколько, выведите любой из них.
Если без сдачи не обойтись, то выведите одно число 0. Если же у Волшебного человечка не хватит денег, чтобы заплатить указанную сумму, выведите одно число –1 (минус один).

