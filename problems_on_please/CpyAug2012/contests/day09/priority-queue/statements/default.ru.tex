% Statements
Реализуйте структуру данных ``очередь с приоритетами'', поддерживающую
следующие операции:

\begin{enumerate}
\item Добавление элемента в очередь.
\item Удаление из очереди элемента с набольшим приоритетом.
\item Изменение приоритета для произвольного элемента, находящегося в
очереди.
\end{enumerate}

\InputFile
Программа получает на вход последовательность команд, по одной команде
в каждой строке. Общее число команд не превосходит
$30\,000$. Команда может иметь один из следующих форматов:

\verb"ADD id priority"~--- добавить в очередь новый элемент
с идентификатором \verb"id" и приоритетом \verb"priority".
Гарантируется, что в очереди нет элемента с таким идентификатором.

\verb"POP"~--- удалить из очереди элемент с наибольшим значением
приоритета. Если таких элементов несколько, то удаляется один
(любой) из них. Гарантируется, что очередь не пуста.

\verb"CHANGE id new_priority"~--- изменить значение приоритета
элемента с идентификатором \verb"id" на значение \verb"new_priority".
Гарантируется, что в очереди есть элемент с таким идентификатором.

Идентификаторы элементов~--- строки, состоящие из строчных латинских
букв длиной не более 10 символов. Идентификаторы~--- произвольные
целые числа.

В самом начале очередь пуста.

\OutputFile
Для каждой команды типа \verb"POP" выведите
идентификатор удаленного элемента и, через пробел,
значение его приоритета.
