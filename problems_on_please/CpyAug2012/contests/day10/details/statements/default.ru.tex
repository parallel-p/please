% Statements
Предприятие «Авто-2012» выпускает двигатели для известных во всём мире автомобилей. Двигатель состоит ровно из $n$ деталей, пронумерованных от 1 до $n$ , при этом деталь с номером $i$ изготавливается за $p_i$ секунд. Специфика предприятия «Авто-2012» заключается в том, что там одновременно может изготавливаться лишь одна деталь двигателя. Для производства некоторых деталей необходимо иметь предварительно изготовленный набор других деталей.

Генеральный директор «Авто-2010» поставил перед предприятием амбициозную задачу --- за наименьшее время изготовить деталь с номером 1, чтобы представить её на выставке.

Требуется написать программу, которая по заданным зависимостям порядка производства между деталями найдёт наименьшее время, за которое можно произвести деталь с номером 1.
\InputFile
Первая строка содержит $n$ ($1 \leq n \leq 100000$) натуральных чисел $p_1, p_2, \dots p_n$, определяющих время изготовления каждой детали в секундах.

Каждая из последующих $n$ строк входного файла описывает характеристики производства деталей. Здесь $i$-ая строка содержит список деталей, которые требуются для производства детали с номером $i$. В списке нет повторяющихся номеров деталей. Список может быть в том числе пустым --- тогда ему будет соответствовать пустая строка! Сумма длин всех списков не превосходит $200000$.

Известно, что не существует циклических зависимостей в производстве деталей.
% Input file description
\OutputFile
В единственной строке выходного файла должно содержаться одно число: минимальное время (в секундах), необходимое для скорейшего производства детали с номером 1.
% Output file description

