Маша и Паша любят играть в разноцветные кубики, причем у каждого из них свой набор, в каждом наборе все кубики различны по цвету. Однажды ребята заинтересовались, сколько существуют цветов таких, что кубики каждого цвета присутствуют в обоих наборах, для этого они занумеровали все цвета случайными числами. На этом их энтузиазм иссяк, поэтому вам предлагается помочь им в оставшейся части.

\InputFile
Номер любого цвета - это число $G$ такое, что $0 \leqslant G \leqslant 10^9$. В первой строке входного файла записаны числа $N$ и $M$ ($0 \leqslant N, M \leqslant 100\,000$) - количество кубиков у Маши и у Паши соответственно. В следующих $N$ строках заданы номера цветов кубиков Маши, в следующих $M$ строках - Паши.

\OutputFile
Выведите первым числом количество, а затем отсортированные по возрастанию номера цветов таких, что кубики каждого цвета есть в обоих наборах, затем количество и отсортированные по возрастанию номера остальных цветов у Маши, потом количество и отсортированные по возрастанию номера остальных цветов у Паши.