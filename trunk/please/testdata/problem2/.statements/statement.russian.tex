%Please contest template file
\documentclass[12pt,a4paper,oneside,twocolumn,landscape]{article}

\usepackage[T2A]{fontenc}
\usepackage[utf8]{inputenc}
\usepackage[english,russian]{babel}
\usepackage[russian,landscape]{olymp}
\usepackage{graphicx}
\DeclareGraphicsRule{*}{mps}{*}{}
\graphicspath{{../statements/}}
\usepackage{amsmath,amssymb,amsthm}
\usepackage{epsfig}
\usepackage{import}
\usepackage{wrapfig}
\usepackage{epigraph}

\contest
{}
{b24}
{30.08.2012}


\renewcommand{\t}[1]{\ifmmode{\mathtt{#1}}\else{\texttt{#1}}\fi}
\renewcommand{\le}{\leqslant}
\renewcommand{\ge}{\geqslant}
\newcommand{\bs}{\mbox{$\backslash$}}

\begin{document}

\raggedbottom

\begin{problem}{b24}{prev.in}{prev.out}{2.1 секунд}{128 мегабайт}{}
\graphicspath{{.././statements/}}


Найдите предыдущую в лексикографическом порядке перестановку. Перестановка вида $N, N - 1, ... , 3, 2, 1$ является предыдущей для $1, 2, 3, ... , N - 1, N$

\InputFile
В первой строке входного файла записано число $N$  ($1 \leqslant N \leqslant 10^5$)  количество элементов в перестановке. Во второй строке записана перестановка.

\OutputFile
В выходной файл вывести $N$ чисел~--- искомую перестановку.


\Example

\begin{example}
\exmp{
3
1 2 3
}{3 2 1 }%
\end{example}


\end{problem}


\end{document}
